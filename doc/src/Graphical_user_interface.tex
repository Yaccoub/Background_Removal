\section{Graphical User Interface}
\label{GUI}

In this chapter, we will present our graphical user interface. We will also discuss the steps required to run the program and visulize these with the help of screenshots.

\medskip

This graphical user interface has been created using the Matlab built-in app developemnt framework App Designer. This framework provides user friendly standard components such as buttons, check boxes, trees, and drop-down lists, that made the usage of our program considerably intuitiv. 

\subsection{Structural Design}

While developing the graphical user interface (GUI) to meet the required  functions, we tried to keep the interface as simple and intuitiv as possible, in order not to overwhelm the user. 

\medskip

Therefore we splited the given specifications into 5 steps and 1 optional. The first being dedicated to selecting paths of the source, background and output directories. If these paths are not valid, an error message will be displayed to the user.

\medskip

The second step is allocated to selecting the configuration parameters variables Left, Right, Start and N. These variable are also subject to a control routine, which verifies their validity.
  
\medskip

Before running the program, the user should select a rendering mode in the third step. The user has a toggle button group containing all 4 possible rendering modes. The use of this toggle box will prevent the user from selecting multiple modes at once. 

\subsection{Interaction Design}

After running the program, the stereo input image stream will be displayed on the left side of the console whereas the rendered output stream will be dispalyed on the right side. To let the user control these streams and be able to play it, pause it or repeat it on looop, a playback control has been developed.

\medskip

As an additional feature, a video with the virtual background will be played in an extra window with the help of the Matlab built-in video player. 








